%! Author = Jan
%! Date = 03.08.2022

% Preamble
\documentclass[fontsize=6pt]{scrreprt}
\RequirePackage[paperheight=100mm,paperwidth=65mm,margin=3mm,heightrounded]{geometry}
\RequirePackage[ngerman]{babel}
\RequirePackage{fontspec}
\RequirePackage{enumitem}
\RequirePackage{standalone}

\setlength{\parindent}{0pt}
\setlength{\parskip}{4pt}
\setmainfont{Arial}

% Document
\begin{document}
  \textbf{Rekrutieren:}
  In einem eigenen Gebiet 1 Einheit + 1 Einheit pro Ausbildungslager hinzufügen.

  \textbf{Entdecken:}
  1 Spielplanplättchen nehmen, an ein eigenes, offenes Gebiet anlegen,
  ggfs. Kreatur erscheinen lassen und aktivieren.

  \textbf{Bauen:}
  In einem eigenen Gebiet 1 Gebäude platzieren und Baukosten bezahlen.
  1 Holz für kleine, 3 Holz für große Gebäude.
  Steine nur auf kleine Felder mit Steinsymbol.
  Nur 1 Gebäudetyp pro Gebiet.

  \textbf{Bewegen:}
  \begin{enumerate}[topsep=1pt, partopsep=0pt, parsep=0pt, itemsep=0pt, leftmargin=12pt]
    \item Beliebig viele Einheiten aus einem eigenen Gebiet in ein benachbartes Gebiet. Kosten:
    \begin{enumerate}[label=\textbf{\arabic*x}, topsep=0pt, partopsep=0pt, parsep=0pt, itemsep=0pt, leftmargin=10pt]
      \item Bewegungspunkt für eine normale Grenze
      \item Bewegungspunkte für eine raue (gelbe) Grenze
    \end{enumerate}
    \item Nach Betreten eines feindlichen Gebiets können diese Einheiten nicht weiter bewegt werden, hier kommt es zum Kampf.
    \item So oft 1. wiederholen bis Bewegungspunkte aufgebraucht - oder Rest verfallen lassen.
    \item Kämpfe abhandeln.
  \end{enumerate}

  \textbf{Kampf:}
  \begin{enumerate}[topsep=1pt, partopsep=0pt, parsep=0pt, itemsep=0pt, leftmargin=12pt]
    \item Zähle Kampfpunkte (1 pro Einheit)
    \item Zähle Karten und Gebäudebonus
    \item Pro Einheit bis zu 1 Nahrung als Kampfpunkt hinzufügen (Angreifer fängt an)
    \item Würfeln mit einem Würfel (Angreifer fängt an)
    \item Der Spieler mit mehr Kampfpunkten gewinnt, bei Gleichstand gewinnt der Verteidiger.
    \item Sterben alle Einheiten eines Spielers ist der Kampf verloren.
    \item Rückzug beliebige eigene oder neutrale Gebiete
  \end{enumerate}

  \textbf{Aufwerten:} 1 Handkarte aktivieren (ohne effekt) oder entfernen (außer Aufruhrkarten), 3 Runen bezahlen für eine Aufwertungskarte auf die Hand.

\end{document}
