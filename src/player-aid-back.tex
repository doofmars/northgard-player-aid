%! Author = Jan
%! Date = 03.08.2022

% Preamble
\documentclass[fontsize=6pt]{scrreprt}
\RequirePackage[paperheight=100mm,paperwidth=65mm,margin=3mm,heightrounded]{geometry}
\RequirePackage[ngerman]{babel}
\RequirePackage{fontspec}
\RequirePackage{enumitem}
\RequirePackage{standalone}

\setlength{\parindent}{0pt}
\setlength{\parskip}{4pt}
\setmainfont{Arial}

% Document
\begin{document}
  \textbf{Handeln (Bei Ernte):} 3 beliebige Ressourcen : 1 Ressource

  \textbf{Spielaufbau:}

  \textbf{Entwicklungskarten:}

  \begin{tabular}{c|cccc}
    Kartenart & 2 Sp & 3 Sp & 4 Sp & 5 Sp \\ \hline
    .      & 4 & 6  & 8  & 10 \\
    ..     & 8 & 12 & 16 & 20 \\
    \ldots & 2 & 3  & 4  & 5  \\
  \end{tabular}

  \textbf{Startresourcen:}
  \begin{itemize}[topsep=1pt, partopsep=0pt, parsep=0pt, itemsep=0pt, leftmargin=12pt]
    \item 2--3 Spieler: 2 Äpfel 2 Holz
    \item 4--5 Spieler: 3 Äpfel 2 Holz
  \end{itemize}

  \textbf{Siegbedingungen:}
  \begin{center}
  Am Jahresende 3 geschlossene Gebiete mit großem Gebäude unter eigener Kontrolle.

  oder

  Die meisten Ruhmpunkte am Spielende.
  \end{center}

  \textbf{Wertung:}
  \begin{itemize}[topsep=1pt, partopsep=0pt, parsep=0pt, itemsep=0pt, leftmargin=12pt]
    \item Ruhmpunkte vom Vorrat
    \item 1 Ruhmpunkt pro 3 beliebige Ressourcen
    \item Ruhmpunkte auf Karten + Wertung der Errungenschaftskarten (variabel)
    \item -5 Ruhmpunkte pro Aufruhrkarte
  \end{itemize}


\end{document}
