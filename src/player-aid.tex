%! Author = Jan
%! Date = 03.08.2022

% Preamble
\documentclass[6pt]{scrreprt}
\usepackage[paperheight=100mm,paperwidth=65mm,margin=4mm,heightrounded]{geometry}
\usepackage[ngerman]{babel}
\usepackage{fontspec}
\usepackage{enumitem}

\setlength{\parindent}{0pt}
\setlength{\parskip}{4pt}
\setmainfont{Arial}

% Packages

% Document
\begin{document}
  \textbf{Entdecken:}
  1 Spielplanplättchen nehmen, an ein eigenes, offenes Gebiet anlegen, ggfs. Kreatur erscheinen lassen und aktivieren.

  \textbf{Rekrutieren:}
  In einem eigenen Gebiet 1 Einheit + 1 Einheit pro Ausbildungslager hinzufügen.

  \textbf{Bauen:}
  In einem eigenen Gebiet 1 Gebäude platzieren und Baukosten bezahlen. 1 Holz für kleine, 3 Holz für große Gebäude. Steine nur auf kleine Felder mit Steinsymbol.

  \textbf{Bewegen:}
  \begin{enumerate}[topsep=1pt, partopsep=0pt, parsep=0pt, itemsep=0pt, leftmargin=12pt]
    \item Beliebig viele Einheiten aus einem eigenen Gebiet in ein benachbartes Gebiet. Kosten:
    \begin{enumerate}[label=\textbf{\arabic*x}, topsep=0pt, partopsep=0pt, parsep=0pt, itemsep=0pt, leftmargin=10pt]
      \item Bewegungspunkt für eine normale Grenze
      \item Bewegungspunkte für eine raue (gelbe) Grenze
    \end{enumerate}
    \item Nach Betreten eines feindlichen Gebiets können diese Einheiten nicht weiter bewegt werden, hier kommt es zum Kampf.
    \item So oft 1. wiederholen bis Bewegungspunkte aufgebraucht - oder Rest verfallen lassen.
    \item Kämpfe abhandeln.
  \end{enumerate}

  \textbf{Kampf:}
  \begin{enumerate}[topsep=1pt, partopsep=0pt, parsep=0pt, itemsep=0pt, leftmargin=12pt]
    \item Zähle Kampfpunkte (1 pro Einheit + evtl. Karten/ Gebäudebonus)
    \item Pro Einheit bis zu 1 Nahrung als Kampfpunkt hinzufügen (Angreifer fängt an)
    \item Würfeln mit einem Würfel (Angreifer fängt an)
    \item Der Spieler mit mehr Kampfpunkten gewinnt, bei Gleichstand gewinnt der Verteidiger.
    \item Sterben alle Einheiten eines Spielers ist der Kampf verloren.
  \end{enumerate}

  \textbf{Handeln (Bei Ernte):} 3 (beliebige) : 1

  \pagebreak

  \textbf{Spielaufbau:}


  \begin{table}[htp]
    \textbf{Entwicklungskarten:}
    \centering
    \label{tab:development_cards}
    \begin{tabular}{c|cccc}
      Kartenart & 2 Sp & 3 Sp & 4 Sp & 5 Sp \\ \hline
      .      & 4 & 6  & 8  & 10 \\
      ..     & 8 & 12 & 16 & 20 \\
      \ldots & 2 & 3  & 4  & 5  \\
    \end{tabular}
  \end{table}

  \textbf{Startresourcen:}
  \begin{itemize}[topsep=1pt, partopsep=0pt, parsep=0pt, itemsep=0pt, leftmargin=12pt]
    \item 2--3 Spieler: 2 Äpfel 2 Holz
    \item 4--5 Spieler: 3 Äpfel 2 Holz
  \end{itemize}

  \textbf{Siegbedingungen:}
  \center
  Am Jahresende 3 geschlossene Gebiete mit großem Gebäude unter eigener Kontrolle.

  oder

  Die meisten Ruhmpunkte am Spielende.
\end{document}
